\documentclass[onecolumn, draftclsnofoot,10pt, compsoc]{IEEEtran}
\usepackage{graphicx}
\usepackage{url}
\usepackage{setspace}
\usepackage{pgfgantt}

\usepackage{geometry}
\geometry{textheight=9.5in, textwidth=7in}

% 1. Fill in these details
\def \CapstoneTeamName{		    Team 25}
\def \CapstoneTeamNumber{		25}
\def \GroupMemberOne{			Vincent Nguyen}
\def \GroupMemberTwo{			Adam Ruark}
\def \GroupMemberThree{			Yihong Liu}
\def \CapstoneProjectName{		HyperRail App}
\def \CapstoneSponsorCompany{	OPEnS Lab}
\def \CapstoneSponsorPerson{	Chet Udell}

% 2. Uncomment the appropriate line below so that the document type works
\def \DocType{	%Problem Statement
				Requirements Document
				%Technology Review
				%Design Document
				%Progress Report
				}
			
\newcommand{\NameSigPair}[1]{\par
\makebox[2.75in][r]{#1} \hfil 	\makebox[3.25in]{\makebox[2.25in]{\hrulefill} \hfill		\makebox[.75in]{\hrulefill}}
\par\vspace{-12pt} \textit{\tiny\noindent
\makebox[2.75in]{} \hfil		\makebox[3.25in]{\makebox[2.25in][r]{Signature} \hfill	\makebox[.75in][r]{Date}}}}
% 3. If the document is not to be signed, uncomment the RENEWcommand below
\renewcommand{\NameSigPair}[1]{#1}

%%%%%%%%%%%%%%%%%%%%%%%%%%%%%%%%%%%%%%%
\begin{document}
\begin{titlepage}
    \pagenumbering{gobble}
    \begin{singlespace}
        \hfill 
        % 4. If you have a logo, use this includegraphics command to put it on the coversheet.
        %\includegraphics[height=4cm]{CompanyLogo}   
        \par\vspace{.2in}
        \centering
        \scshape{
            \huge CS Capstone \DocType \par
            {\large\today}\par
            \vspace{.5in}
            \textbf{\Huge\CapstoneProjectName}\par
            \vfill
            {\large Prepared for}\par
            \Huge \CapstoneSponsorCompany\par
            \vspace{5pt}
            {\Large\NameSigPair{\CapstoneSponsorPerson}\par}
            {\large Prepared by }\par
            Group\CapstoneTeamNumber\par
            % 5. comment out the line below this one if you do not wish to name your team
            %\CapstoneTeamName\par 
            \vspace{5pt}
            {\Large
                \NameSigPair{\GroupMemberOne}\par
                \NameSigPair{\GroupMemberTwo}\par
                \NameSigPair{\GroupMemberThree}\par
            }
            \vspace{20pt}
        }
        \begin{abstract}
        % 6. Fill in your abstract
        
        
        \end{abstract}     
    \end{singlespace}
\end{titlepage}
\newpage
\pagenumbering{arabic}
\tableofcontents

\clearpage

% Revisions Table
\section{Summarized Changes}
\medskip
\begin{centering}

\begin{tabular}{|p{1in} | p{2.5in} | p{2.5in}|}
\hline
\textbf{Section} & \textbf{Original} & \textbf{New}\\\hline
Title Page & Three team members. & Yihong left the team. Updated abstract. \\\hline

Limitations & HyperRail is limited to areas with low interference. & Personal computer must be near the HyperRail to connect and interact with it.\\\hline

Database Requirements & Database requires safe and secure user login and will also store configurations. & Database is stored locally on the user's machine. Configurations are stored locally on the user's machine.\\\hline

Reliability & No direct connection to the hardware. & Direct connection to the hardware is allowed.\\\hline

Efficiency & The application should support multiple users. & Removed. \\\hline

Integrity & The application should display or log what user triggered which action. & The application will notify the user whether an action was successful or failed and display a server log message. \\\hline

Portability & Mobile devices and personal computers can connect access the web application. & Personal computers can host and access the web application and can interact with the hardware by connecting to its network. \\\hline

Safety & The system will only allow registered users to log in to interact with the application. & Removed. \\\hline

Gantt Chart & Chart displays planned progression. & Chart matches actual progression. \\\hline
\end{tabular}

\end{centering}

\pagebreak


\section{Introduction}

\subsection{Purpose}
The purpose of this document is to provide an overview of the product to be delivered to the client by the end of Spring 2019. The client has listed a number of features, or project requirements, that they wish the capstone team to implement.


\subsection{Scope}
The hardware for the HyperRail project has already been decided upon and its code has been mostly developed. For this project, the hardware code needs to be optimized and a web-based graphical user interface and central server are needed to complete the software portion of the HyperRail application. The specified features are meant to allow a user to remotely communicate with a specified HyperRail system, which is handled and abstracted through the central server. The features have two levels of priority: required and convenient. Specifications that are required are needed for the system to be fully developed and deployed. Those that are convenient are stretch goals that are not required, but would be nice to have to make the system more robust.



\section{Product Overview}

\subsection{Product Description}
The HyperRail is a small railway where an automated environmental sensor package, which contains a variety of different sensors, can traverse through a space and collect information as it travels along the railway. Using the HyperRail application, the sensor package's settings can be customized and monitored. The application currently allows the user to specify the speed of the package, current length of the rail, and the size of the spool used for the motors. It also monitors the position of the sensor package along the railway by calculating the number of motor steps it has taken and updates the position display in real-time. However, it currently requires a direct connection, wired or wireless, to the sensor package itself, limiting the HyperRail's use.


\subsection{User Characteristics}
Users of the HyperRail will be using it to monitor environmental changes. They can monitor levels of carbon dioxide, moisture, light levels, and other variables in the environment depending on the sensors used in the package. No programming experience is required, but the user will need to know how to interact with computers as the application abstracts the programming required to automate the HyperRail system. The user can interpret the data collected in any manner they would like.


\subsection{Limitations}
The use of the HyperRail is limited to areas with low interference. Because a computer communicates with the sensor package using a radio or Bluetooth connection, there cannot be strong interference disrupting the connection or the data collected by the sensor package will be lost. The HyperRail is also limited to safe or contained environments. Because the HyperRail system utilizes a physical setup, the railway or the sensor package can be damaged by entities in the environment. The computer and sensor package also requires electricity to function. Therefore, the HyperRail is limited to areas with power.



\section{Functional Requirements}

\subsection{Usability Requirements}
This application must have a user interface that is intuitive to use. This means that the primary functionality of the app such as defining the parameters of the system's journey and uploading them to the robot should never be more than one click away. Secondary functionality such as saving and loading configurations can be a few clicks away. This is to provide a readable interface that is not cluttered. 

The application must also provide feedback to the user when an action succeeds or fails. This can be done with a pop-up containing a status message for the last completed action.

\subsection{Database Requirements}
There are some attributes that must be stored in a database. First, users must be allowed to log in, therefore users and their encrypted passwords must be stored in a database. Second, we potentially want to allow saved configurations and these must be saved and loaded; this will also require a database. 

\subsection{Code Quality}
The underlying code that will be running this application must be maintainable for teams in the future to improve upon. This means that the code must be homogeneous and well documented. These qualities will allow the any portion of code to be understandable just from looking at its source and to have the reason for its existence documented.

\subsection{Reliability} 
The application needs to allow the user to operate the HyperRail system remotely, using the custom-developed web interface to finish their specific request without directly connecting to the micro-controller.

\subsection{Efficiency}
The application needs to be able to handle the different commands from multiple users and handle those commands accurately and quickly. The server must be able to handle multiple users connected at once and ensure that their user experience is not affected by any other connected users. User actions must be handled in a timely fashion.

\subsection{Integrity}
Integrity for this system is defined by how well it can track those commands from different users then handle those commands successfully. This means that user actions must be accountable by either displaying what user triggered each action on a status check, or by storing a log of previous actions performed on an individual system.

\subsection{Flexibility}
The application should account for future changes. If functions are modified or added to the application, version control should be used to ensure that there is working code.

\subsection{Portability}
This application should be able to interact and configure new sensor packages without major changes to the software. The software itself should be portable in the sense that mobile devices and personal computers can access the web application and interact with a specified HyperRail system. %It should be able to be report into a new operating system environment without taking a lot of rework. I would say java programming language could be best for this kind of software.

\subsection{Safety}
This application will need to utilize a secure connection to ensure that all content is controlled and private. The system will also only allow registered users to log in and interact with the system.


\subsection{Performance Requirements}
The essential measurement standard for the performance requirement is if the application is able to satisfy all the requirements and include all of the functionality needed to be deployed. Although incorporating the stretch goals is not mandatory for the application's release, it will make the application more robust. %If the application can handle the requirement more accurate (For example, the sponsor can be delivered to the more accurate location) or more efficient (the Hyperrail system can get the command and respond faster) than the user through the mic-controller, then our application totally meets the high-level performance requirements.


\section{Stretch Goals}
One stretch goal is to allow the sensor package to interact with existing actuators in the space. Because actuators are machines that perform simple movements on objects, such as valves or switches, this interaction will give the application additional capabilities depending on the application of the actuators.

Another stretch goal is the capability to configure a fleet of sensor systems to use the same configuration. This allows for quick configuration of a group of sensor packages that are intended for the same purpose.

Another feature that we would like to include is the capability to save and load configurations to allow users to share their settings across devices. This would make the application even more portable for users and allow the sensor system to be truly configured from anywhere. It would also allow the user to experiment with different configurations while preserving the old settings.



\section{Gantt Chart}
\begin{center}
	\begin{ganttchart}{1}{6}
		\gantttitle{Months}{6} \\
		\gantttitlelist{1,2,3,4,5,6}{1} \\
		\ganttbar{Create Database for HyperRail}{1}{2} \ganttnewline
		\ganttbar{Design and Test Web Interface}{2}{5} \ganttnewline
		\ganttbar{Optimize Code}{4}{5} \ganttnewline
		\ganttbar{Work on Stretch Goals}{5}{6} \ganttnewline
		\ganttbar{Documentation and Wrap Up}{5}{6}
	\end{ganttchart}
\end{center}


%  My ideas (Adam) on what needs to get completed. Change as needed. Wednesday night whatever is listed below should get written up.

% * Create a user-friendly web page
% * Host the website somewhere
%     * Notify someone if site goes down
% * Translate what the user input into valid configuration for Feather M0
% * Create database to store users and configurations
% * New configurations for new sensors and new techniques (stopping at different points depending on the sensor)
% * Maintainable code (testable, readable)

% * Also allow wired connection straight to device. This means native application, not a web app (will need to discuss).

% Stretch Goals:
% * Allow users to configure mutliple devices at once
% * Save/load configurations
% * Actuator interactions


\end{document}